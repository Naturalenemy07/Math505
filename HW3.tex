\documentclass{article}
\usepackage[utf8]{inputenc}
\usepackage[papersize={8.5in,11in},margin=0.8in]{geometry}
\usepackage{xcolor}
\usepackage{color, colortbl}
\usepackage{graphicx}
\usepackage{tikz}
\usepackage{amssymb}
\definecolor{LightCyan}{rgb}{0.88,1,1}
\definecolor{LightRed}{rgb}{1,0.49,0.49}

\title{MATH 505 HW 3}
\author{John Caruthers}
\date\today

\begin{document}
\maketitle

\begin{itemize}
    \item[Exp. 2] Use a truth table to show that an argument is valid.
    
    (a) Let \emph{p} be the statement "I will buy a skateboard." Let \emph{q} be the statement "I will buy a scooter." Write the following argument in symbolic form.\\
    \hspace*{1cm} I will buy a skateboard, or I will buy a scooter\\
    \hspace*{1cm} I will not buy a skateboard\\
    \begin{tikzpicture}
    \hspace*{1cm}
    \draw[densely dashed] (0,0)--(5,0);
    \end{tikzpicture}\\
    \hspace*{1cm} Therefore, I will buy a scooter.
    
    \hspace*{1cm} {\color{blue} \emph{p} $\vee$ \emph{q}}\\
    \hspace*{1cm} {\color{blue} $\sim$\emph{p}}\\
    \begin{tikzpicture}
    \hspace*{1cm}
    {\color{blue} \draw[densely dashed] (0,0)--(1,0);}
    \end{tikzpicture}\\
    \hspace*{1cm} {\color{blue} $\therefore$ \emph{p}}
    
    (b) Consider the symbolic pattern for the argument from (a). (An argument that follows this pattern is known classically as a \emph{disjunctive syllogism.}) Complete the truth table, which shows that the argument is valid.
    
    \begin{center}
        \begin{tabular}{|c|c|c|c|c|c|}
            \hline
            \emph{p} & \emph{q} & $\sim$\emph{p} & \emph{p} $\vee$ \emph{q} & (\emph{p} $\vee$ \emph{q}) $\wedge$ $\sim$\emph{q} & \emph{q}\\
            \hline
            T & T & F & T & F & T\\
            \hline
            T & F & F & T & T & F \\
            \hline
            F & T & T & T & F & T \\
            \hline
            F & F & T & F & F & F \\
            \hline
        \end{tabular}
    \end{center}
    
    \begin{center}
        \begin{tabular}{|c|c|c|c|c|c|}
            \hline
            \emph{p} & \emph{q} & $\sim$\emph{p} & \emph{p} $\to$ \emph{q} & (\emph{p} $\to$ \emph{q}) $\wedge$ $\sim$\emph{q} & $\sim$\emph{q}\\
            \hline
        \end{tabular}
    \end{center}

\end{itemize}
\end{document}

