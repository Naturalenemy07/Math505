\documentclass{article}
\usepackage[utf8]{inputenc}
\usepackage[papersize={8.5in,11in},margin=0.8in]{geometry}
\usepackage{adjustbox}
\usepackage{xcolor}
\usepackage{color, colortbl}
\usepackage{graphicx}
\definecolor{LightCyan}{rgb}{0.88,1,1}
\usepackage{tikz}
\usepackage{amssymb}

\title{MATH 505 Assignment 2}
\author{John Caruthers}
\date\today

\begin{document}
\maketitle

\begin{itemize}
    \item[1.] Decide whether the arguments are valid. Explain.
    
    (a) All math is fun.\\
    \hspace*{0.6cm}Logic is math.\\
    \begin{tikzpicture}
    \hspace*{0.6cm}
    \draw[densely dashed] (0,0)--(2.5,0);
    \end{tikzpicture}\\
    \hspace*{0.6cm}Logic is fun.
    
    \hspace*{0.6cm}{\color{blue} This is valid by rule of detachment.}
    
    (b) If I exercise, then I will be tired.\\
    \hspace*{0.6cm}I do not exercise.\\
    \begin{tikzpicture}
    \hspace*{0.6cm}
    \draw[densely dashed] (0,0)--(2.5,0);
    \end{tikzpicture}\\
    \hspace*{0.6cm}Therefore, I am not tired.
    
    \hspace*{0.6cm}{\color{blue} This argument is invalid by falsity of inverse.}
    
    (c) All natural numbers are integers.\\
    \hspace*{0.55cm}All integers are rational numbers.\\
    \hspace*{0.6cm}$\pi$ is not a rational number.\\
    \begin{tikzpicture}
    \hspace*{0.6cm}
    \draw[densely dashed] (0,0)--(2.5,0);
    \end{tikzpicture}\\
    \hspace*{0.6cm}$\pi$ is not an integer.
    
    \hspace*{0.6cm}{\color{blue} This is valid by rule of contrapositive}
    
    (d) All multiples of 2 are even.\\
    \hspace*{0.6cm}18 is a multiple of 3\\
    \begin{tikzpicture}
    \hspace*{0.6cm}
    \draw[densely dashed] (0,0)--(2.5,0);
    \end{tikzpicture}\\
    \hspace*{0.6cm}18 is even.
    
    \hspace*{0.6cm}{\color{blue} It is valid because the conclusion is true when the conjunction of the premises are true .}
    
    \item[2.] In the book, the \textbf{Rule of Detachment} and the \textbf{Chain Rule} have accompanying truth tables.  In these truth tables, they mark the rows which justify these two rules.  Do the same for \textbf{Rule of Contrapositive}.
    
    The rule of contrapositive states:\\
    \hspace*{0.6cm}$p \to q$\\
    \hspace*{0.6cm}$\sim q$\\
    \begin{tikzpicture}
    \hspace*{0.6cm}
    \draw[densely dashed] (0,0)--(2.5,0);
    \end{tikzpicture}\\
    \hspace*{0.6cm}$\therefore$ $ \sim p$
    
    \begin{centering}
        \begin{tabular}{|c|c|c|c|c|c|}
            \hline
            $p$ & $q$ & $\sim q$ & $p \to q$ & $p \to q$ $\wedge$ $\sim q$ & $\sim p$\\
            \hline
            T & T & F & T & F & F\\
            \hline
            T & F & T & F & F & F\\
            \hline
            F & T & F & T & F & T\\
            \hline
            \rowcolor{LightCyan}
            F & F & T & T & T & T\\
            \hline
        \end{tabular}
    \end{centering}
    
    {\color{blue} The Rule of Contrapositive is valid because in all instances where the conjunction of the premises is true, the conclusion is also true.}
\end{itemize}

\end{document}
