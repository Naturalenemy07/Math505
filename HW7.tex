\documentclass{article}
\usepackage[utf8]{inputenc}
\usepackage[papersize={8.5in,11in},margin=0.8in]{geometry}
\usepackage{amsfonts}
\usepackage{amsmath}
\usepackage{amssymb}
\usepackage{xcolor}
\usepackage{color, colortbl}

\title{Math 505 HW7}
\author{John Caruthers}
\date\today

\begin{document}
\maketitle

Prove by induction.
\begin{itemize}
    \item[1.] For all natural numbers $n, 3+6+9+...+3n=\frac{3n(n+1)}{2}$
    \begin{itemize}
        \item[]\emph{Basis:} $n=1$, so $3=\frac{3(1+1)}{2}=\frac{3(2)}{2}=\frac{6}{2}=3$
        
        \item[]\emph{Inductive Hypothesis:} $n=k$, so $3+6+9+...+3k=\frac{3k(k+1)}{2}$
        
        \item[]\emph{Inductive Step:} $n=k+1$, so $3+6+9+...+3k+3(k+1)=\frac{3(k+1)((k+1)+1)}{2}$
        \begin{align}
            3+6+9+...+3k&=\frac{3k(k+1)}{2}\nonumber\\
            \frac{3k(k+1)}{2}+3(k+1)&=\frac{3(k+1)((k+1)+1)}{2}\nonumber\\
            \frac{3k(k+1)}{2}+\frac{2\cdot3(k+1)}{2}&=\frac{3(k+1)((k+1)+1)}{2}\nonumber\\
            \frac{3k(k+1)+6(k+1)}{2}&=\frac{3(k+1)((k+1)+1)}{2}\nonumber\\
            \frac{3k^2+9k+9}{2}&=\frac{3k^2+9k+9}{2}\nonumber
        \end{align}
        By induction, $n\in \mathbb{Z},3+6+9+...+3n=\frac{3n(n+1)}{2}$. $\Box$
    \end{itemize}
    
    \item[2.] For all positive integers $n$, $1^2+2^2+3^2+...+n^2=\frac{n(n+1)(2n+1)}{6}$.
    \begin{itemize}
        \item[]\emph{Basis:} $n=1$, so $1^2=\frac{1(1+1)(2+1)}{6}=\frac{6}{6}=1$
        
        \item[]\emph{Inductive Hypothesis:} $n=k$, so $1^2+2^2+3^2+...+k^2=\frac{k(k+1)(2k+1)}{6}$
        
        \item[]\emph{Inductive Step:} $n=k+1$ so $1^2+2^2+3^2+...+k^2+(k+1)^2=\frac{(k+1)(k+2)(2k+3)}{6}$
        \begin{align}
            \frac{k(k+1)(2k+1)}{6}+(k+1)^2&=\frac{(k+1)(k+2)(2k+3)}{6}\nonumber\\
            \frac{k(k+1)(2k+1)+6(k+1)^2}{6}&=\frac{(k+1)(k+2)(2k+3)}{6}\nonumber\\
            \frac{2k^3+k^2+2k^2+k+6(k^2+2k+1)}{6}&=\frac{(k+1)(k+2)(2k+3)}{6}\nonumber
        \end{align}
        Solve left side of equation, then see if equal to right side of equation.
        \begin{align}
            \frac{2k^3+9k^2+13k+6}{6}&=\frac{(k^2+2k+k+2)(2k+3)}{6}\nonumber\\
            \frac{2k^3+9k^2+13k+6}{6}&=\frac{2k^3+3k^2+4k^2+6k+2k^2+3k+4k+6}{6}\nonumber\\
            \frac{2k^3+9k^2+13k+6}{6}&=\frac{2k^3+9k^2+13k+6}{6}\nonumber
        \end{align}
        By induction, For all positive integers $n$, $1^2+2^2+3^2+...+n^2=\frac{n(n+1)(2n+1)}{6}$. $\Box$
    \end{itemize}
    \newpage
    \item[3.] For all natural numbers $1^3+2^3+3^3+...+n^3=\Big[\frac{n(n+1)}{2}\Big]^2$
    \begin{itemize}
        \item[]\emph{Basis:} $n=1$, so $1^3=\Big[\frac{1(2)}{2}\Big]^2=1^2=1$
        
        \item[]\emph{Inductive Hypothesis:} $n=k$, so $1^3+2^3+3^3+...+k^3=\Big[\frac{k(k+1)}{2}\Big]^2$
        
        \item[]\emph{Inductive Step:} $n=k+1$, so $1^3+2^3+3^3+...+k^3+(k+1)^3=\Big[\frac{(k+1)(k+2)}{2}\Big]^2$
        \begin{align}
            \Big[\frac{k(k+1)}{2}\Big]^2+(k+1)^3&=\Big[\frac{(k+1)(k+2)}{2}\Big]^2\nonumber\\
            \frac{k(k+1)k(k+1)}{4}+\frac{4(k+1)^3}{4}&=\Big[\frac{(k+1)(k+2)}{2}\Big]^2\nonumber\\
            \frac{(k^2+k)(k^2+k)+4(k+1)^3}{4}&=\Big[\frac{(k+1)(k+2)}{2}\Big]^2\nonumber\\
            \frac{k^4+2k^3+k^2+4(k+1)(k+1)(k+1)}{4}&=\Big[\frac{(k+1)(k+2)}{2}\Big]^2\nonumber\\
            \frac{k^4+6k^3+13k^2+12k+4}{4}&=\Big[\frac{(k+1)(k+2)}{2}\Big]^2\nonumber
        \end{align}
        Now, solve for the right side of the equation to see if it is equal
        \begin{align}
            \frac{k^4+6k^3+13k^2+12k+4}{4}&=\frac{(k^2+2k+k+2)^2}{4}\nonumber\\
            \frac{k^4+6k^3+13k^2+12k+4}{4}&=\frac{k^4+3k^3+2k^2+3k^3+9k^2+6k+2k^2+6k+4}{4}\nonumber\\
            \frac{k^4+6k^3+13k^2+12k+4}{4}&=\frac{k^4+6k^3+13k^2+12k+4}{4}\nonumber
        \end{align}
        By induction, for all natural numbers $1^3+2^3+3^3+...+n^3=\Big[\frac{n(n+1)}{2}\Big]^2$. $\Box$
    \end{itemize}
    
    \item[4.] For all positive integers $n$, $1+4+7+...+(3n-2)=\frac{n(3n-1)}{2}$
    \begin{itemize}
        \item[]\emph{Basis:} $n=1$, so $1=\frac{1(3-1)}{2}=\frac{1(2)}{2}=1$
        
        \item[]\emph{Inductive Hypothesis:} $n=k$, so $1+4+7+...+(3k-2)=\frac{k(3k-1)}{2}$
        
        \item[]\emph{Inductive Step:} $n=k+1$, so $1+4+7+...+(3k-2)+(3(k+1)-2)=\frac{(k+1)(3(k+1)-1)}{2}$
        \begin{align}
            \frac{k(3k-1)}{2}+(3(k+1)-2)&=\frac{(k+1)(3(k+1)-1)}{2}\nonumber\\
            \frac{3k^2-k+(2(3k+1))}{2}&=\frac{(k+1)(3(k+1)-1)}{2}\nonumber\\
            \frac{3k^2-k+(6k+2)}{2}&=\frac{(k+1)(3(k+1)-1)}{2}\nonumber\\
            \frac{3k^2+5k+2}{2}&=\frac{(k+1)(3(k+1)-1)}{2}\nonumber
        \end{align}
        Next, solve for the right side of the equation to see if it is equal
        \begin{align}
            \frac{3k^2+5k+2}{2}&=\frac{(k+1)(3k+3-1)}{2}\nonumber\\
            \frac{3k^2+5k+2}{2}&=\frac{(k+1)(3k+2)}{2}\nonumber\\
            \frac{3k^2+5k+2}{2}&=\frac{3k^2+2k+3k+2}{2}\nonumber\\
            \frac{3k^2+5k+2}{2}&=\frac{3k^2+5k+2}{2}\nonumber
        \end{align}
        By induction, for all positive integers $n$, $1+4+7+...+(3n-2)=\frac{n(3n-1)}{2}$. $\Box$
    \end{itemize}
    \newpage
    \item[5.] For all positive integers $n$, $n<2^n$
    \begin{itemize}
        \item[]\emph{Basis:} $n=1$, so $1<2^1$ which is $1<2$
        
        \item[]\emph{Inductive Hypothesis:} $n=k$, so $k<2^k$
        
        \item[]\emph{Inductive Step:} $n=k+1$, so $(k+1)<2^{k+1}$. 
        \begin{center}
            Since $2^{k+1}=2\cdot2^k$, we can write: $(k+1)<2\cdot2^k$.\\
            This means that $(k+1)$ is 2 times what $k$ is: $2^k$
        \end{center}
        By induction, for all positive integers $n$, $n<2^n$. $\Box$
    \end{itemize}
    
    \item[7.] For all positive integers $n$, $2$ divides $3^n-1$.\\
    We will use two theorems that have been proven true and don't need to be proven true here.  First is: {\color{blue} An integer $n$ is \textbf{even} provided that there is some integer $k$ with $n=2k$}. The second is: {\color{purple} If $x$ and $y$ are even integers, then $x+y$ is even.}.
    \begin{itemize}
        \item[]\emph{Basis:} $n=1$ so $2$ divides $3^1-1=3-1=2$
        
        \item[]\emph{Inductive Hypothesis:} $n=k$ so $2$ divides $3^k-1$
        
        \item[]\emph{Inductive Step:} $n=k+1$ so $2$ divides $3^{k+1}-1$.
        \begin{align}
            3^{k+1}-1&=3\cdot(3^{k}-1)\nonumber\\
            3^{k+1}-1&=(2+1)\cdot(3^{k}-1)\nonumber\\
            3^{k+1}-1&=2\cdot3^k+(3^k-1)\nonumber
        \end{align}
        By theorems stated above, $2\cdot3^k$ is an even number, therefore it is divisible by $2$.  By the Inductive Hypothesis, $3^k-1$ is divisible by $2$, so it is even.  Therefore, $2\cdot3^k+(3^k-1)$ is an even number and by induction: for all positive integers $n$, $2$ divides $3^n-1$. $\Box$
    \end{itemize}
    
    \item[9.] Experiment with several small values of $n$, make a conjecture about a formula, and prove the conjecture for the following sum:
    \begin{center}
        $\frac{1}{1\cdot2}+\frac{1}{2\cdot3}+\frac{1}{3\cdot4}+...+\frac{1}{n(n+1)}$
        
    \renewcommand{\arraystretch}{1.4}
    \begin{tabular}{|c|c|c|}
    
        \hline
        $n$ & $denominator$ & $sum$ \\
        \hline
        1 & $\frac{1}{2}$ & $\frac{1}{2}$\\
        \hline
        2 & $\frac{1}{6}$ & $\frac{2}{3}$\\
        \hline
        3 & $\frac{1}{12}$ & $\frac{3}{4}$\\
        \hline
        4 & $\frac{1}{20}$ & $\frac{4}{5}$\\
        \hline
        5 & $\frac{1}{30}$ & $\frac{5}{6}$\\
        \hline
    \end{tabular}
    \end{center}
    
    Conjecture: For positive integers $n$, the sum: $\frac{1}{1\cdot2}+\frac{1}{2\cdot3}+\frac{1}{3\cdot4}+...+\frac{1}{n(n+1)}=\frac{n}{n+1}$
    \begin{itemize}
        \item[]\emph{Basis:} $n=1$, so $\frac{1}{1\cdot2}=\frac{1}{2}$
        
        \item[]\emph{Inductive Hypothesis:} $n=k$, so $\frac{1}{1\cdot2}+\frac{1}{2\cdot3}+\frac{1}{3\cdot4}+...+\frac{1}{k(k+1)}=\frac{k}{k+1}$
        
        \item[]\emph{Inductive Step:} $n=k+1$, so $\frac{1}{1\cdot2}+\frac{1}{2\cdot3}+\frac{1}{3\cdot4}+...+\frac{1}{k(k+1)}+\frac{1}{(k+1)(k+2)}=\frac{k+1}{k+2}$
        \begin{align}
            \frac{k}{k+1}+\frac{1}{(k+1)(k+2)}&=\frac{k+1}{k+2}\nonumber\\
            \frac{k(k+2)}{(k+1)(k+2)}+\frac{1}{(k+1)(k+2)}&=\frac{k+1}{k+2}\nonumber\\
            \frac{k(k+2)+1}{(k+1)(k+2)}&=\frac{k+1}{k+2}\nonumber\\
            \frac{k^2+2k+1}{(k+1)(k+2)}&=\frac{k+1}{k+2}\nonumber
        \end{align}
        $k^2+2k+1$ can be simplified to $(k+1)(k+1)$
        \begin{align}
            \frac{(k+1)(k+1)}{(k+1)(k+2)}&=\frac{k+1}{k+2}\nonumber\\
            \frac{k+1}{k+2}&=\frac{k+1}{k+2}\nonumber
        \end{align}
        By induction, For positive integers $n$, the sum: $\frac{1}{1\cdot2}+\frac{1}{2\cdot3}+\frac{1}{3\cdot4}+...+\frac{1}{n(n+1)}=\frac{n}{n+1}$. $\Box$
    \end{itemize}
\end{itemize}

\end{document}
