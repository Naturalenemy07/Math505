\documentclass{article}
\usepackage[utf8]{inputenc}
\usepackage[papersize={8.5in,11in},margin=0.8in]{geometry}
\usepackage{tabto}
\usepackage{xcolor}
\usepackage{color, colortbl}

\title{MATH 505 HW 2}
\author{John Caruthers}
\date\today

\begin{document}
\maketitle

\begin{itemize}
    \item [Exp 1.(a)] Write each of these conditional statements in the form "If \emph{p}, then \emph{q}.":
    
    (1) Hai can vote if he is 18. {\color{blue} If he is 18, then Hai can vote.}
    
    (2) A muddy path implies that I will wear my boots. {\color{blue} If the path is muddy, then I will wear my boots.}
    
    (3) When I finish studying, then I will go to the party. {\color{blue} If I finish studying, then I will go to the party.}
    
    (4) Number \emph{x} is a whole number only if \emph{x} is an integer. {\color{blue} If \emph{x} is an integer, then number \emph{x} is a whole number.}
    
    \item [Exp 1.(b)] We know that the contrapositive "If $\sim$\emph{q}, then $\sim$\emph{p}" is equivalent to "If \emph{p}, then \emph{q}." Use the fact and the work we did in (a) to write a statement equivalent to each statement in (a).
    
    (1) If Hai can't vote, then he isn't 18.
    
    (2) If I don't wear my boots, then the path isn't muddy.
    
    (3) If I don't go to the party, then I didn't finish studying.
    
    (4) If the number \emph{x} isn't a whole number, then \emph{x} isn't an integer
    
    \item [Exp 2.] One of De Morgan's Laws states that for any statements \emph{p} and \emph{q}, $\sim$(\emph{p}$\vee$\emph{q}) is equivalent to ($\sim$\emph{p})$\wedge$($\sim$\emph{q}).
    
    (a) Identify statements \emph{p} and \emph{q} so that the following statement can be understood as "\emph{p} or \emph{q}.": Amelia made an A in mathematics, or Amelia made a B in English.\\
    \emph{p}: Amelia made an A in mathematics\\
    \emph{q}: Amelia made a B in English
    
    (b) One way to write the negation of the statement in (a) is: It is not true that Amelia made an A in mathematics or Amelia made a B in English.  Write the negation another way using De Morgan's Laws.\\
    $\sim$(\emph{p} $\vee$ \emph{q}) is equivalent to $\sim$\emph{p} $\wedge$ $\sim$\emph{q} therefore we could negate the statement in (a) by {\color{blue} Amelia didn't make in A in mathematics and she didn't make a B in English.}
    
    (c) Use one of De Morgan's Laws to write the negation of this statement: Ardie can pay me now, or Ardie will work for me later.\\
    \emph{p}: Ardie can pay me now.\\
    \emph{q}: Ardie will work for me later.\\
    $\sim$(\emph{p} $\vee$ \emph{q}) is equivalent to $\sim$\emph{p} $\wedge$ $\sim$\emph{q} therefore to negate the statements, we can write: {\color{blue} Ardie can't pay me now and Ardie can't work for me later.}
    
    \item[1.] Let A represent "It is snowing"; B, "The roofs are white."; C, "The streets are slick"; and D, "The trees are covered with ice." Write the following in symbolic notation. 

\end{itemize}
\end{document}
