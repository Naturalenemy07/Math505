\documentclass{article}
\usepackage[utf8]{inputenc}
\usepackage[papersize={8.5in,11in},margin=0.8in]{geometry}
\usepackage{xcolor}

\title{MATH 505 HW 2}
\author{John Caruthers}
\date\today

\begin{document}
\maketitle

\begin{itemize}
    \item [Exp 1.(a)] Write each of these conditional statements in the form "If \emph{p}, then \emph{q}.":
    
    (1) Hai can vote if he is 18. {\color{blue} If he is 18, then Hai can vote.}
    
    (2) A muddy path implies that I will wear my boots. {\color{blue} If the path is muddy, then I will wear my boots.}
    
    (3) When I finish studying, then I will go to the party. {\color{blue} If I finish studying, then I will go to the party.}
    
    (4) Number \emph{x} is a whole number only if \emph{x} is an integer. {\color{blue} If \emph{x} is an integer, then number \emph{x} is a whole number.}
    
    \item [Exp 1.(b)] We know that the contrapositive "If $\sim$\emph{q}, then $\sim$\emph{p}" is equivalent to "If \emph{p}, then \emph{q}." Use the fact and the work we did in (a) to write a statement equivalent to each statement in (a).
    
    (1) If Hai can't vote, then he isn't 18.
    
    (2) If I don't wear my boots, then the path isn't muddy.
    
    (3) If I don't go to the party, then I didn't finish studying.
    
    (4) If the number \emph{x} isn't a whole number, then \emph{x} isn't an integer
    
    \item [Exp 2.] One of De Morgan's Laws states that for any statements \emph{p} and \emph{q}, $\sim$(\emph{p}$\vee$\emph{q}) is equivalent to ($\sim$\emph{p})$\wedge$($\sim$\emph{q}).
    
    (a) Identify statements \emph{p} and \emph{q} so that the following statement can be understood as "\emph{p} or \emph{q}.": Amelia made an A in mathematics, or Amelia made a B in English.\\
    \emph{p}: Amelia made an A in mathematics\\
    \emph{q}: Amelia made a B in English
    
    (b) One way to write the negation of the statement in (a) is: It is not true that Amelia made an A in mathematics or Amelia made a B in English.  Write the negation another way using De Morgan's Laws.\\
    $\sim$(\emph{p} $\vee$ \emph{q}) is equivalent to $\sim$\emph{p} $\wedge$ $\sim$\emph{q} therefore we could negate the statement in (a) by {\color{blue} Amelia didn't make in A in mathematics and she didn't make a B in English.}
    
    (c) Use one of De Morgan's Laws to write the negation of this statement: Ardie can pay me now, or Ardie will work for me later.\\
    \emph{p}: Ardie can pay me now.\\
    \emph{q}: Ardie will work for me later.\\
    $\sim$(\emph{p} $\vee$ \emph{q}) is equivalent to $\sim$\emph{p} $\wedge$ $\sim$\emph{q} therefore to negate the statements, we can write: {\color{blue} Ardie can't pay me now and Ardie can't work for me later.}
    
    \item[1.] Let A represent "It is snowing"; B, "The roofs are white."; C, "The streets are slick"; and D, "The trees are covered with ice." Write the following in symbolic notation. 
    
    (a) If it is snowing, then the trees are covered with ice. {\color{blue} A $\to$ D}
    
    (b) If it is not snowing, then the roofs are not white. {\color{blue} $\sim$A $\to$ $\sim$B}
    
    (c) If the streets are not slick, then it is not snowing. {\color{blue} $\sim$C $\to$ $\sim$A}
    
    (d) If the streets are slick, then the trees are not covered with ice. {\color{blue} C $\to$ $\sim$D}
    
    \item[2.] In Exercise 1, assume that A is true, that B and C are both false, and that D is true.  Classify the conditional statement as either true or false.
    
    (a) \begin{center}
        \begin{tabular}{|c|c|c|}
            \hline
            A & D & A $\to$ D \\
            \hline
            T & T & T\\
            \hline
        \end{tabular}
    \end{center}
    
    (b) \begin{center}
        \begin{tabular}{|c|c|c|c|c|}
             \hline
             A & B & $\sim$A & $\sim$B & $\sim$A $\to$ $\sim$B\\
             \hline
             T & F & F & T & T \\
             \hline
        \end{tabular}
    \end{center}
    
    (c) \begin{center}
        \begin{tabular}{|c|c|c|c|c|}
            \hline
            C & A & $\sim$C & $\sim$A & $\sim$C $\to$ $\sim$A \\
            \hline
            F & T & T & F & F \\
            \hline
        \end{tabular}
    \end{center}
    
    (d) \begin{center}
        \begin{tabular}{|c|c|c|c|c|}
            \hline
            C & D & $\sim$D & C $\to$ $\sim$D \\
            \hline
            F & T & F & T \\
            \hline
        \end{tabular}
    \end{center}
    
    \item[3.] State the \textcolor{blue}{converse}, the \textcolor{olive}{inverse}, and the \textcolor{purple}{contrapositive} of each of the following conditionals.
    
    (a) If a triangle is a right triangle, then one angle has a measurement of 90$^\circ$.\\
    Converse: {\color{blue} If one angle of a triangle has a measurement of 90$^\circ$, then it is a right triangle.}\\
    Inverse: {\color{olive} If a triangle isn't a right triangle, then one angle doesn't have a measurement of 90$^\circ$.}\\
    Contrapositive: {\color{purple} If a triangle doesn't have one angle with a measurement of 90$^\circ$, then that triangle isn't a right triangle.}
    
    (b) If a number is a prime, then it is odd.\\
    Converse: {\color{blue} If a number is odd, then it is a prime number.}\\
    Inverse: {\color{olive} If a number isn't prime, then it isn't odd.}\\
    Contrapositive: {\color{purple} If a number isn't odd, then it isn't a prime number.}
    
    (c) If two lines are parallel, then alternate interior angles are equal.\\
    Converse: {\color{blue} If the alternate interior angles are equal, then the two lines are parallel.}\\
    Inverse: {\color{olive} If two lines are not parallel, then the alternate interior angles are not equal.}\\
    Contrapositive: {\color{purple} It the alternate interior angles are not equal, then the two lines are not parallel.}
    
    (d) If Joyce is smiling, then she is happy.\\
    Converse: {\color{blue} If she is happy, then Joyce is smiling.}\\
    Inverse: {\color{olive} If Joyce isn't smiling, then she isn't happy}\\
    Contrapositive: {\color{purple} If she isn't happy, then Joyce isn't smiling.}
    
    \item[4.] Find the truth value of each of the following statements.
    
    \item[5.] Write the converse, inverse, and contrapositive of each statement in Exercise 4.

\end{itemize}
\end{document}
