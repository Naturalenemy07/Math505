\documentclass{article}
\usepackage[utf8]{inputenc}
\usepackage[papersize={8.5in,11in},margin=0.8in]{geometry}
\usepackage{xcolor}
\usepackage{color, colortbl}
\usepackage{graphicx}
\definecolor{LightCyan}{rgb}{0.88,1,1}
\definecolor{LightRed}{rgb}{1,0.49,0.49}

\title{MATH 505 HW 2}
\author{John Caruthers}
\date\today

\begin{document}
\maketitle

\begin{itemize}
    \item [Exp 1.(a)] Write each of these conditional statements in the form "If \emph{p}, then \emph{q}.":
    
    (1) Hai can vote if he is 18. {\color{blue} If he is 18, then Hai can vote.}
    
    (2) A muddy path implies that I will wear my boots. {\color{blue} If the path is muddy, then I will wear my boots.}
    
    (3) When I finish studying, then I will go to the party. {\color{blue} If I finish studying, then I will go to the party.}
    
    (4) Number \emph{x} is a whole number only if \emph{x} is an integer. {\color{blue} If \emph{x} is an integer, then number \emph{x} is a whole number.}
    
    \item [Exp 1.(b)] We know that the contrapositive "If $\sim$\emph{q}, then $\sim$\emph{p}" is equivalent to "If \emph{p}, then \emph{q}." Use the fact and the work we did in (a) to write a statement equivalent to each statement in (a).
    
    (1) If Hai can't vote, then he isn't 18.
    
    (2) If I don't wear my boots, then the path isn't muddy.
    
    (3) If I don't go to the party, then I didn't finish studying.
    
    (4) If the number \emph{x} isn't a whole number, then \emph{x} isn't an integer
    
    \item [Exp 2.] One of De Morgan's Laws states that for any statements \emph{p} and \emph{q}, $\sim$(\emph{p}$\vee$\emph{q}) is equivalent to ($\sim$\emph{p})$\wedge$($\sim$\emph{q}).
    
    (a) Identify statements \emph{p} and \emph{q} so that the following statement can be understood as "\emph{p} or \emph{q}.": Amelia made an A in mathematics, or Amelia made a B in English.\\
    \hspace*{1cm}\emph{p}: Amelia made an A in mathematics\\
    \hspace*{1cm}\emph{q}: Amelia made a B in English
    
    (b) One way to write the negation of the statement in (a) is: It is not true that Amelia made an A in mathematics or Amelia made a B in English.  Write the negation another way using De Morgan's Laws.\\
    $\sim$(\emph{p} $\vee$ \emph{q}) is equivalent to $\sim$\emph{p} $\wedge$ $\sim$\emph{q} therefore we could negate the statement in (a) by {\color{blue} Amelia didn't make in A in mathematics and she didn't make a B in English.}
    
    (c) Use one of De Morgan's Laws to write the negation of this statement: Ardie can pay me now, or Ardie will work for me later.\\
    \hspace*{1cm}\emph{p}: Ardie can pay me now.\\
    \hspace*{1cm}\emph{q}: Ardie will work for me later.\\
    $\sim$(\emph{p} $\vee$ \emph{q}) is equivalent to $\sim$\emph{p} $\wedge$ $\sim$\emph{q} therefore to negate the statements, we can write: {\color{blue} Ardie can't pay me now and Ardie can't work for me later.}
    
    \item[1.] Let A represent "It is snowing"; B, "The roofs are white."; C, "The streets are slick"; and D, "The trees are covered with ice." Write the following in symbolic notation. 
    
    (a) If it is snowing, then the trees are covered with ice. {\color{blue} A $\to$ D}
    
    (b) If it is not snowing, then the roofs are not white. {\color{blue} $\sim$A $\to$ $\sim$B}
    
    (c) If the streets are not slick, then it is not snowing. {\color{blue} $\sim$C $\to$ $\sim$A}
    
    (d) If the streets are slick, then the trees are not covered with ice. {\color{blue} C $\to$ $\sim$D}
    
    \item[2.] In Exercise 1, assume that A is true, that B and C are both false, and that D is true.  Classify the conditional statement as either true or false.
    
    (a) \begin{center}
        \begin{tabular}{|c|c|c|}
            \hline
            A & D & A $\to$ D \\
            \hline
            T & T & T\\
            \hline
        \end{tabular}
    \end{center}
    
    (b) \begin{center}
        \begin{tabular}{|c|c|c|c|c|}
             \hline
             A & B & $\sim$A & $\sim$B & $\sim$A $\to$ $\sim$B\\
             \hline
             T & F & F & T & T \\
             \hline
        \end{tabular}
    \end{center}
    
    (c) \begin{center}
        \begin{tabular}{|c|c|c|c|c|}
            \hline
            C & A & $\sim$C & $\sim$A & $\sim$C $\to$ $\sim$A \\
            \hline
            F & T & T & F & F \\
            \hline
        \end{tabular}
    \end{center}
    
    (d) \begin{center}
        \begin{tabular}{|c|c|c|c|c|}
            \hline
            C & D & $\sim$D & C $\to$ $\sim$D \\
            \hline
            F & T & F & T \\
            \hline
        \end{tabular}
    \end{center}
    
    \item[3.] State the \textcolor{blue}{converse}, the \textcolor{olive}{inverse}, and the \textcolor{purple}{contrapositive} of each of the following conditionals.
    
    (a) If a triangle is a right triangle, then one angle has a measurement of 90$^\circ$.\\
    Converse: {\color{blue} If one angle of a triangle has a measurement of 90$^\circ$, then it is a right triangle.}\\
    Inverse: {\color{olive} If a triangle isn't a right triangle, then one angle doesn't have a measurement of 90$^\circ$.}\\
    Contrapositive: {\color{purple} If a triangle doesn't have one angle with a measurement of 90$^\circ$, then that triangle isn't a right triangle.}
    
    (b) If a number is a prime, then it is odd.\\
    Converse: {\color{blue} If a number is odd, then it is a prime number.}\\
    Inverse: {\color{olive} If a number isn't prime, then it isn't odd.}\\
    Contrapositive: {\color{purple} If a number isn't odd, then it isn't a prime number.}
    
    (c) If two lines are parallel, then alternate interior angles are equal.\\
    Converse: {\color{blue} If the alternate interior angles are equal, then the two lines are parallel.}\\
    Inverse: {\color{olive} If two lines are not parallel, then the alternate interior angles are not equal.}\\
    Contrapositive: {\color{purple} If the alternate interior angles are not equal, then the two lines are not parallel.}
    
    (d) If Joyce is smiling, then she is happy.\\
    Converse: {\color{blue} If she is happy, then Joyce is smiling.}\\
    Inverse: {\color{olive} If Joyce isn't smiling, then she isn't happy}\\
    Contrapositive: {\color{purple} If she isn't happy, then Joyce isn't smiling.}
    
    \item[4.] Find the truth value of each of the following statements.
    
    (a) If the moon orbits the earth, then $2 + 2 = 5$.\\
    \hspace*{1cm}\emph{p}: The moon orbits the earth\\
    \hspace*{1cm}\emph{q}: $2 + 2 = 5$\\
        \hspace*{1cm}\begin{tabular}{|c|c|c|}
            \hline
            \emph{p} & \emph{q} & \emph{p} $\to$ \emph{q} \\
            \hline
            T & F & F\\
            \hline
        \end{tabular}
    
    \hspace*{1cm}{\color{purple} The truth value of this statement if False}
    
    (b) If Lincoln was the first United States president, then $2 + 2 = 5$.\\
    \hspace*{1cm}\emph{p}: Lincoln was the first United States president\\
    \hspace*{1cm}\emph{q}: $2 + 2 = 5$\\
        \hspace*{1cm}\begin{tabular}{|c|c|c|}
            \hline
            \emph{p} & \emph{q} & \emph{p} $\to$ \emph{q} \\
            \hline
            F & F & T\\
            \hline
        \end{tabular}
    
    \hspace*{1cm}{\color{blue} The truth value of this statement if True}
    
    (c) If Mexico is the southern neighbor of the United States, then $3 + 3 = 6$.\\
    \hspace*{1cm}\emph{p}: Mexico is the southern neighbor of the United States\\
    \hspace*{1cm}\emph{q}: $3 + 3 = 6$\\
        \hspace*{1cm}\begin{tabular}{|c|c|c|}
            \hline
            \emph{p} & \emph{q} & \emph{p} $\to$ \emph{q} \\
            \hline
            T & T & T\\
            \hline
        \end{tabular}
    
    \hspace*{1cm}{\color{blue} The truth value of this statement if True}
    \newpage
    
    (d) If Canada is in Asia, then $3 + 3 = 6$.\\
    \hspace*{1cm}\emph{p}: Canada is in Asia\\
    \hspace*{1cm}\emph{q}: $3 + 3 = 6$\\
        \hspace*{1cm}\begin{tabular}{|c|c|c|}
            \hline
            \emph{p} & \emph{q} & \emph{p} $\to$ \emph{q} \\
            \hline
            F & T & T\\
            \hline
        \end{tabular}
    
    \hspace*{1cm}{\color{blue} The truth value of this statement if True}
    
    \item[5.] Write the converse, inverse, and contrapositive of each statement in Exercise 4.
    
    (a) If the moon orbits the earth, then $2 + 2 = 5$.\\
    Converse: {\color{blue} If $2 + 2 = 5$, then the moon orbits the earth.}\\
    Inverse: {\color{olive} If the moon doesn't orbit the earth, then $2 + 2 \neq 5$.}\\
    Contrapositive: {\color{purple} If $2 + 2 \neq 5$, then the moon doesn't orbit the earth.}
    
    (b) If Lincoln was the first United States president, then $2 + 2 = 5$.\\
    Converse: {\color{blue} If $2 + 2 = 5$, the Lincoln was the first United States president.}\\
    Inverse: {\color{olive} If Lincoln wasn't the first United states president, then $2 + 2 \neq 5$.}\\
    Contrapositive: {\color{purple} If $2 + 2 \neq 5$, the Lincoln wasn't the first United States president.}
    
    (c) If Mexico is the southern neighbor of the United States, then $3 + 3 = 6$.\\
    Converse: {\color{blue} If $3 + 3 = 6$, then Mexico is the southern neighbor of the United States.}\\
    Inverse: {\color{olive} If Mexico isn't the southern neighbor of the United States, then $3 + 3 \neq 6$.}\\
    Contrapositive: {\color{purple} If $3 + 3 \neq 6$, then Mexico isn't the southern neighbor of the United States.}
    
    (d) If Canada is in Asia, then $3 + 3 = 6$.\\
    Converse: {\color{blue} If $3 + 3 = 6$, the Canada is in Asia.}\\
    Inverse: {\color{olive} If Canada isn't in Asia, then $3 + 3 \neq 6$.}\\
    Contrapositive: {\color{purple} If $3 + 3 \neq 6$, then Canada isn't in Asia.}
    
    \item[6.] Using the notation of Exercise 1, translate the following into sentences:
    
    (a) $\sim$A $\to$ B {\color{blue} If it isn't snowing, then the roofs are white.}
    
    (b) $\sim$C $\to$ $\sim$B {\color{blue} If the streets are not slick, then the roofs are not white.}
    
    (c) ($\sim$B $\wedge$ $\sim$C) $\to$ A {\color{blue} If the roofs are not white and the streets are not slick, then it is snowing.}
    
    (d) (A $\vee$ B) $\to$ $\sim$C {\color{blue} If it is snowing or the roofs are white, then the streets are not slick.}
    
    (e) $\sim$(A $\to$ $\sim$B) is equivalent to A $\wedge$ B {\color{blue} It is snowing and the roofs are white.}
    
    (f)  ($\sim$C $\vee$ $\sim$A) $\to$ $\sim$B is equivalent to $\sim$(C $\wedge$ A) $\to$ $\sim$B {\color{blue} If the streets are not slick and it isn't snowing, then the roofs are not white.}
    
    \item[7.] In the game show "Truth of Despair", the contestants are presented with two doors.  Behind one door there is a new car; behind the other door there is a skateboard.  On each door there is a sign.  Contestants are told that exactly one of the two signs is true.  Explain carefully how to determine which door leads to the new car from these signs: 
    
    The sign on the first door reads, \emph{"If there is a car behind this door, then there is a skateboard behind the other door."} and the sign on the second door reads \emph{"There is a car behind this door"}.
    
    {\color{blue} Using symbolic notation, the sign on Door 1 reads: D1 $\wedge$ C $\to$ D2 $\wedge$ S. Following the constraints of the game, both statements on Door 1 need to be either all true or false; shown in the light cyan rows below.  This means that the conditional on Door 1 can only be true, otherwise it would break the constraints of the game.}
    \begin{center}
        \begin{tabular}{|c|c|c|}
            \hline
            D1 $\wedge$ C & D2 $\wedge$ S & D1 $\wedge$ C $\to$ D2 $\wedge$ S \\
            \hline
            \rowcolor{LightCyan}
            T & T & T\\
            \hline
            \rowcolor{LightRed}
            T & F & F\\
            \hline
            \rowcolor{LightRed}
            F & T & T\\
            \hline
            \rowcolor{LightCyan}
            F & F & T\\
            \hline
        \end{tabular}
    \end{center}
    
    {\color{blue} Using symbolic notation, the sign on Door 2 reads: D2 $\wedge$ C. Since the sign on Door 1 must be true, the sign on Door 2 must be false.  This means that Door 1 has the car and Door 2 has the skateboard.}
    
    \begin{center}
        \begin{tabular}{|c|c|c|c|c|c|}
            \hline
            D1 $\wedge$ C $\to$ D2 $\wedge$ S & D2 $\wedge$ C & D1 $\wedge$ C & D1 $\wedge$ S & D2 $\wedge$ C & D2 $\wedge$ S\\
            \hline
            T & F & T & F & F & T\\
            \hline
        \end{tabular}
    \end{center}

    \item[9.] Commercials and advertisements are often based on the assumption that naive audiences will accept the converse, inverse, contrapositive , and original statements are all true.  Write the converse, inverse, and contrapositive of the following statements.  Think about whether they have the same truth values.
    
    (a) If you brush your teeth with White-as-Snow, then you have fewer cavities.\\
    Converse: {\color{blue} If you have fewer cavities, then you brush your teeth with White-as-Snow.}\\
    Inverse: {\color{olive} If you don't brush your teeth with White-as-Snow, then you will won't have fewer cavities.}\\
    Contrapositive: {\color{purple} If you don't have fewer cavities, then you don't brush your teeth with White-as-Snow.}
    
    (b) If you like this book, then you love mathematics.\\
    Converse: {\color{blue} If you love mathematics, then you like this book.}\\
    Inverse: {\color{olive} If you don't like this book, then you don't love mathematics.}\\
    Contrapositive: {\color{purple} If you don't love mathematics, then you don't like this book.}
    
    (c) If you are strong, then you eat Barlies for breakfast.\\
    Converse: {\color{blue} If you eat Barlies for breakfast, then you are strong.}\\
    Inverse: {\color{olive} If you are not strong, then you don't eat Barlies for breakfast.}\\
    Contrapositive: {\color{purple} If you don't eat Barlies for breakfast, the you are not strong.}
    
    (d) If you use Wave, then your clothes are bright and colorful.\\
    Converse: {\color{blue} If your clothes are bright and colorful, then you use Wave.}\\
    Inverse: {\color{olive} if you don't use Wave, then your clothes are not bright and colorful.}\\
    Contrapositive: {\color{purple} If you clothes are not bright and colorful, then you don't use wave.}
    
    \item[10.] From an insurance policy: If the recomputed premium exceeds the premium stated on the declaration page, then you must pay the excess to Guarantee Auto.
    
    (a) Suppose the recomputed premium is \$800 while the premium on the declaration page is \$600.  What must the person do?\\
    {\color{blue} The person must pay the excess (\$200) to Guarantee Auto.}
    
    (b) Suppose the recomputed premium is \$550 while the premium on the declaration page is \$600.  What does the policy statement indicate that the insured person must do?\\
    {\color{blue} Since the recomputed amount doesn't exceed the premium on the declaration page, then the person doesn't need to do anything.}
    
    \item[11.] From IRS instructions: if your return is more than 60 days late, the minimum penalty will be \$100 or the amount of any tax you owe, whichever is smaller.
    
    (a) Suppose the return is 65 days late and you owe \$75 in taxes.  What is the minimum penalty?\\
    {\color{blue} \$75}
    
    (b) Suppose the return is 59 days late.  What does the policy say about the penalty?\\
    {\color{blue} Since the return isn't more than 60 days late, the is no penalty.}
    
    \item[12.] Write the following in "if-then" form:
    
    (a) Carelessness leads to accidents. {\color{blue} If you are careless, then it will lead to accidents.}
    
    (b) Whenever I see June, my heart throbs. {\color{blue} If I see June, then my heart throbs.}
    
    (c) A triangle is isosceles if two sides are congruent. {\color{blue} If a triangle is isosceles, then two sides are congruent.}
    
    (d) I will be happy if I pass. {\color{blue} If I pass, then I will be happy.}
    
\end{itemize}
\end{document}
